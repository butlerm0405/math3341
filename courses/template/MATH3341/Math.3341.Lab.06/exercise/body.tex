If you haven't downloaded and unzipped \href{https://libaoj.in/courses/2021f/MATH3341/zip/Math.3341.zip}{\texttt{Math.3341.zip}}. Download and unzip it under \verb|H:| (H Drive if you are working on the Remote Lab). Change the current working directory by typing \verb|cd H:\Math.3341\Math.3341.Lab.06| in the Command Window, and type \verb|edit lab_06_script| in the Command Window to edit \verb|lab_06_script.m|.

%---------------------------------------------
\section{Solve a System with LU Decomposition}
%---------------------------------------------
\label{sec:lu}
\begin{enumerate}[(a)]
\item Define matrix \verb|A| and vector \verb|b| as \eqref{eq:ls}.
  \begin{equation}
  \label{eq:ls}
  \underbrace{
  \begin{bmatrix}
    7 & -26 &  45 & -47 \\
    1 &   2 &   3 &   4 \\
    2 & -11 & -12 & -13 \\
    4 & -17 &  30 &  35 \\
  \end{bmatrix}}_{A}
  \underbrace{
  \begin{bmatrix}
    x_1 \\
    x_2 \\
    x_3 \\
    x_4 \\
  \end{bmatrix}}_{\mathbf{x}}
  =
  \underbrace{
  \begin{bmatrix}
     -98 \\
      30 \\
    -108 \\
     200 \\
  \end{bmatrix}}_{\mathbf{b}}
  \end{equation}
\item Calculate the LU decomposition \verb|L|, \verb|U| of the matrix \verb|A|.
\item \label{enu:lz} Solve the following system \eqref{eq:lz} and store the solution to \verb|z|.
  \begin{equation}
  \label{eq:lz}
  L \mathbf{z} = \mathbf{b}.
  \end{equation}
\item \label{enu:ux} Then solve the following system \eqref{eq:ux} and store the solution to \verb|x|.
    \begin{equation}
    \label{eq:ux}
    U \mathbf{x} = \mathbf{z}.
    \end{equation}
\item Check your solution by calculating the norm of the residual $\|A\mathbf{x} - \mathbf{b}\|_2$ and store the result to \verb|res|.
\end{enumerate}
%---------------------------------------------
\section{Varying the Vector $\mathbf{b}$}
%---------------------------------------------
Suppose we want to solve the system for each integer value of $m$ in between $m = 0$ and $m = 20$. This time use the LU decomposition of the system matrix; perform the decomposition only once and use the lower and upper triangular factors repeatedly to find each successive solution. Then generate a table (Table \ref{tab:solution}) and a plot (Figure \ref{fig:solution}) of the solution versus the integer $m$.
\begin{equation}
  \label{eq:sys}
  \begin{cases}
    3 x + y + z = m \\
    x - 5 y + 2 z = 5 \\
    2 x + y + 5 z = 10 \\
  \end{cases}
\end{equation}
To do this you'll follow the steps below:
\begin{enumerate}[(a)]
\item Define coefficient matrix \verb|A| given in \eqref{eq:sys}, and get the LU decomposition \verb|L|, \verb|U| of the matrix \verb|A|.
\item Define a vector \verb|m| which ranges from $0$ to $20$ with step size $1$.
\item Then create a for-loop, of which the loop iterator \verb|i| starts from \verb|1| to \verb|length(m)|. In the body of the loop, define a column vector \verb|b| as the right-hand side of \eqref{eq:sys}, where $m$ should be the \verb|i|th component of \verb|m|. Then repeat \eqref{enu:lz} and \eqref{enu:ux} in Part \ref{sec:lu}. Store the solution \verb|x| to the \verb|i|th row of \verb|X|.
\item Format the output of \verb|m| and \verb|X| to a file called \verb|solution.tex| as you did in Part 3 of Lab 05:
  \begin{enumerate}[(i)]
    \item Use \verb|fprintf| to print out the setup for the \emph{table} and \emph{tabular} environments. The first column of the table is centered while the rest three columns are right-justified in \LaTeX{}.
    \item Between \verb|\toprule| and \verb|\midrule|, use \verb|fprintf| to print out the heading of the table. The column widths are $4, 11, 11, 11$, respectively.
    \item Between \verb|\midrule| and \verb|\bottomrule|, use a for-loop to print each row of the table. Note that the $i$th row of the table consists of the \verb|i|th component of \verb|m| and the \verb|i|th row of matrix \verb|X|. The column widths are $2, 9, 9, 9$, respectively. For floating point numbers, output 6 digits after the decimal point.
    \item Call \verb|type('solution.tex')| to print the content of \verb|solution.tex|.
  \end{enumerate}
\item Plot the solution versus $m$ using a for-loop as you did in Part 4 of Lab 05:
  \begin{enumerate}[(i)]
    \item Get the size of \verb|X| and assign it to \verb|XSize|. Define a cell array \verb|styles|, of which the entries are dashed line with hexagram, dotted line with pentagram, solid line with diamond.
    \item The use a for-loop to plot each column of \verb|X| versus \verb|m| in the same figure window with the above styles.
    \item Add labels, title, grid, legend as shown in Figure \ref{fig:solution}.
    \item Save the plot to a file named \verb|lab_06_plot.pdf|.
  \end{enumerate}
\end{enumerate}
%---------------------------------------------
Type \verb|diary('lab_06_output.txt')| in the Command Window, run the script file \verb|lab_06_script.m|, and type \verb|diary off| in the Command Window. Upload \verb|lab_06_output.txt|, \verb|lab_06_script.m|, \verb|solution.tex|, and \verb|lab_06_plot.pdf| to the folder \verb|src| on Overleaf.

On Overleaf, open \verb|body.tex| under the folder \verb|LaTeX|. In the last section of the report, you will reproduce Section \ref{sec:bol} using \LaTeX{}. You may find the following helpful:

\begin{itemize}
  \item You may use enviroments such as  \verb|equation|, \verb|cases|, \verb|figure|, and \verb|table|.
  \item You may use \verb|\includegraphics[width=amount unit]{/path/to/figure.pdf}| to specify the width of a figure. In our case, the width of the figure is \verb|0.85\textwidth|.
  \item You may use \verb|\ref{labelName}| to refer to figures, tables; use \verb|\eqref{labelName}| to refer to equations.
  \item For special symbols, you may look them up in \href{https://libaoj.in/files/LaTeX.Mathematical.Symbols.pdf}{\LaTeX{}.Mathematics.Symbols.pdf}.
  \item You may use \verb|\begin{table}[!hbtp]
\end{table}
| to include the table you got from MATLAB.
\end{itemize}

Recompile and submit the PDF file generated by Overleaf to WyoCourses.

\newpage
%---------------------------------------------
\section{Basics of \LaTeX{}}
\label{sec:bol}
%---------------------------------------------
\subsection{LU Decomposition}

Given the linear system \eqref{eq:varyRHS}
\begin{equation}
  \label{eq:varyRHS}
  \begin{cases}
    3x + y + z = m \\
    x - 5y + 2z = 5 \\
    2x + y + 5z = 10
  \end{cases}
\end{equation}
where $m = 0, 1, 2, \ldots, 20$. Using LU Decomposition we can obtain the solution to the linear system \eqref{eq:varyRHS} for corresponding $m$ (see Table \ref{tab:solution} and Figure \ref{fig:solution}).

\begin{table}[!hbtp]
\end{table}


\begin{figure}[!hbtp]
  \centering
  \includegraphics[width=0.85\textwidth]{../Math.3341.Lab.06.ans/lab_06_plot.pdf}
  \caption{Solution to the linear system vs. $m$}
  \label{fig:solution}
\end{figure}
