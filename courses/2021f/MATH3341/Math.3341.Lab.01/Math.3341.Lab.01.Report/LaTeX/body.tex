\section{Output}
\lstinputlisting[style=Plain]{../src/lab_01_output.txt}

\newpage
% \begin{lstlisting}[style=Plain]
\begin{verbatim}
\section{Basics of \LaTeX{}}

\subsection{Simplifying Fractions}
Consider the function
$$
f(x) = \frac{x^2 - 1}{x + 1}.
$$
To simplify this funciton we can factor the numerator and cancel like terms
\begin{align*}
    f(x)
       & = \frac{x^2 - 1}{x + 1}
    \\ & = \frac{(x - 1) (x + 1)}{x + 1}
    \\ & = x - 1.
\end{align*}

\subsection{Matrix}
A general $3 \times 3$ matrix $A$ has the form
$$
A =
\begin{bmatrix}
    1 & 2 & 3 \\
    4 & 5 & 6 \\
    7 & 8 & 9 \\
\end{bmatrix}.
$$

\subsection{The Millennium Prize Problems}
\emph{The Millennium Prize Problems} are seven problems in mathematics that were stated by
the \textbf{Clay Mathematics Institute} on May 24, 2000. The problems are
\begin{enumerate}[(1)]
  \item Birch and Swinnerton-Dyer conjecture,
  \item Hodge conjecture,
  \item Navier–Stokes existence and smoothness,
  \item P versus NP problem,
  \item Poincar\'{e} conjecture,
  \item Riemann hypothesis,
  \item Yang–Mills existence and mass gap.
\end{enumerate}
\textsc{The Riemann zeta function} is defined for complex $s$ with real part greater than $1$
by the absolutely convergent infinite series
$$
\zeta(s) = \sum_{n=1}^{\infty} \frac{1}{n^s} = \frac{1}{1^s} + \frac{1}{2^s} + \frac{1}{3^s} + \cdots
$$
The practical uses of the Riemann hypothesis include many propositions known true under the
Riemann hypothesis, and some that can be shown to be equivalent to the Riemann hypothesis:
\begin{itemize}
  \item Distribution of prime numbers,
  \item Growth of arithemtic functions,
  \item Large prime gap conjecture,
  \item Criteria equivalent to the Riemann hypothesis.
\end{itemize}
\end{verbatim}
% \end{lstlisting}
