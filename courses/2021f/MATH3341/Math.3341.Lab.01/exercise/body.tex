In this lab you will practice using the command window to carry out basic calculations, learn how to use the help function, MATLAB documentation, and record your input and output using the \verb`diary` function. First, follow the Windows Intructions to create the working directory.

%---------------------------------------------
\section{Arithmetics in MATLAB}
%---------------------------------------------
%---------------------------------------------
\begin{enumerate}[(1)]
    \item In the command window enter the command \verb`diary('lab_01_output.txt')` this will create a \verb`.txt` file. This will record all input and output in the command window until you type the command \verb`diary off`.
    \item Type the command \verb`beep off`. This will disable the sound that plays when there is an error in your code.
    \item Use the \verb`help functionName` to search for the relevant function in the problems below. Consider your `answer' to this question to be the output generated by the \verb`help` command.
    \begin{enumerate}
         \item What type of logarithm does the \verb`log(x)` function calculate?
         \item Does the MATLAB default setting calculate trigonometric functions in radians or degrees?
     \end{enumerate}
     \item \label{enum:4}Carry out the following calculations using normal math operators:
        \begin{enumerate}
          \item \label{enum:a} $2+5$
          \item \label{enum:b} $4^5$
          \item \label{enum:c} $7\cdot 6$
          \item \label{enum:d} $3/8$
          \item \label{enum:e} $54460-2342$
          \item \label{enum:f} $\cos(50^{\circ})$
          % built in function commands
          \item \label{enum:g} $\sqrt{4}$
          \item \label{enum:h} $\ln(3)$
          \item \label{enum:i} $\sin\left(\frac{\pi}{2}\right)$
          \item \label{enum:j} $e^{34}$
        \end{enumerate}
      \item For questions \ref{enum:a} -- \ref{enum:e}, you must also carry out each calculation using functional notation for each operation. Use the \verb`help` command and/or search the MATLAB documentation to find what the functional notation is for each operation.
    \item When you complete the above tasks enter the command \verb`diary off`. This will stop recording the input and output in the command window.
\end{enumerate}
Follow the Overleaf Instructions to set up an account, make a copy of the template, then upload \verb`lab_01_output.txt` to the folder \verb`src` on Overleaf. Next open \verb|body.tex| under the folder \verb|LaTeX|. In the last section of the report, you will reproduce the following using \LaTeX{}.
Once you finish, recompile, and submit the generated \verb|.pdf| file to WyoCourses.

%---------------------------------------------
\section{Basics of \LaTeX{}}
%---------------------------------------------
\subsection{Simplifying Fractions}
Consider the function
$$
f(x) = \frac{x^2 - 1}{x + 1}.
$$
To simplify this funciton we can factor the numerator and cancel like terms
\begin{align*}
    f(x)
       & = \frac{x^2 - 1}{x + 1}
    \\ & = \frac{(x - 1) (x + 1)}{x + 1}
    \\ & = x - 1.
\end{align*}

\subsection{Matrix}
A general $3 \times 3$ matrix $A$ has the form
$$
A =
\begin{bmatrix}
    1 & 2 & 3 \\
    4 & 5 & 6 \\
    7 & 8 & 9 \\
\end{bmatrix}.
$$

\subsection{The Millennium Prize Problems}
\emph{The Millennium Prize Problems} are seven problems in mathematics that were stated by the \textbf{Clay Mathematics Institute} on May 24, 2000. The problems are
\begin{enumerate}[(1)]
  \item Birch and Swinnerton-Dyer conjecture,
  \item Hodge conjecture,
  \item Navier–Stokes existence and smoothness,
  \item P versus NP problem,
  \item Poincaré conjecture,
  \item Riemann hypothesis,
  \item Yang–Mills existence and mass gap.
\end{enumerate}
\textsc{The Riemann zeta function} is defined for complex $s$ with real part greater than $1$ by the absolutely convergent infinite series
$$
\zeta(s) = \sum_{n=1}^{\infty} \frac{1}{n^s} = \frac{1}{1^s} + \frac{1}{2^s} + \frac{1}{3^s} + \cdots
$$
The practical uses of the Riemann hypothesis include many propositions known true under the Riemann hypothesis, and some that can be shown to be equivalent to the Riemann hypothesis:
\begin{itemize}
  \item Distribution of prime numbers,
  \item Growth of arithemtic functions,
  \item Large prime gap conjecture,
  \item Criteria equivalent to the Riemann hypothesis.
\end{itemize}
