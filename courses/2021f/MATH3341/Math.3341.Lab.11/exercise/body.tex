If you haven't downloaded and unzipped \href{https://libaoj.in/courses/2021f/MATH3341/zip/Math.3341.zip}{\texttt{Math.3341.zip}}. Download and unzip it under \verb|H:| (H Drive if you are working on the Remote Lab). Change the current working directory by typing \verb|cd H:\Math.3341\Math.3341.Lab.11| in the Command Window, and type \verb|edit lab_11_script| in the Command Window to edit \verb|lab_11_script.m|.

\section{Built-in Integration Functions}
\begin{enumerate}[(a)]
\item Use both \verb|polyint| and \verb|integral| to evaluate $\displaystyle \int_{-1}^{3} (x^2 - 2x + 1) \, dx$.
    \begin{enumerate}[(1)]
        \item Define the lower bound \verb|a| and upper bound \verb|b|.
        \item Define the polynomial \verb|P| as $p(x) = x^{2} - 2x + 1$.
        \item Integrate $p(x) = x^{2} - 2x + 1$ using \verb|polyint| and store the result to \verb|pI|.
        \item By fundamental theorem of calculus, evaluate the integral \verb|pI| on $[a, b]$ using \verb|polyval| and store the result to \verb|pI_value|.
        \item Define the anonymous function \verb|f| by $f(x) = x^2 - 2x + 1$, and then use \verb|integral| to evaluate $\displaystyle \int_{-1}^{3} (x^2 - 2x + 1) \, dx$ and store it to \verb|I|.
    \end{enumerate}
\item Evaluate the previous integral again, now using \verb|trapz| and \verb|cumtrapz|.
\item Use \verb|integral2| to evaluate $\displaystyle \int_{-\pi}^{-3\pi / 2} \int_{0}^{2 \pi} (y \sin{x} + x \cos{y}) \, dy \, dx$.
\item Use \verb|integral3| to evaluate $\displaystyle \int_{0}^{1} \int_{x^2}^{x} \int_{x-y}^{x+y} y \, dz \, dy \, dx$.
\end{enumerate}
\section{Gauss Quadrature}
\begin{enumerate}[(a)]
    \item Implement Gauss quadrature using $n$ Gauss nodes, which is given by Equation \eqref{eq:gauss}, in the function file \verb|gauss_quad.m|.
        \begin{equation}
            \label{eq:gauss}
            \int_{-1}^{1} f(x) \, dx \approx \sum_{i=1}^{n} w_i f(x_i).
        \end{equation}
    \item Use \verb|gauss_quad| to evaluate the integral
        $$
        \int_{1}^{1.6} \frac{2x}{x^2 - 4} \, dx,
        $$
        with $n = 1, 2, \ldots , 15$ Gauss nodes.

        Note: \verb|legendre_pair.m| is provided to calculate $x_i$ and $w_i$. Use \verb|help legendre_pair| to check the usage.
\end{enumerate}

At last, call \verb|diary('lab_11_output.txt')|, run the scripts \verb|lab_11_script.m|, then call \verb|diary off|. You will upload the script files \verb|lab_11_script.m|, \verb|lab_11_output.txt|, and \verb|gauss_quad.m| to Overleaf. Then recompile, and submit the generated .pdf file on WyoCourses.
